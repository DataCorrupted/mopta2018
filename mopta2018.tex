%**************
% Preamble
%**************
\documentclass[8pt]{beamer}

\usepackage{pifont }
\expandafter\def\expandafter\insertshorttitle\expandafter{%  
    \insertshorttitle\hfill%  
    \insertframenumber\,/\,\inserttotalframenumber}  
 
 
% Set style
% Include packages
\usepackage{ marginnote}
\usepackage[normalem]{ulem}


\usepackage{amsmath,amsfonts,amssymb,latexsym,amsthm,
                         graphicx,graphics,float,subfigure,epsf,enumerate,hyperref,algorithm,algorithmic,multirow,framed}
  
% THEOREM Environments ---------------------------------------------------
%\newtheorem{theorem}{Theorem}[section]
% \newtheorem{corollary}[theorem]{Corollary}
% \newtheorem{lemma}[theorem]{Lemma}
  \newtheorem{assumption}[theorem]{Assumption}
 \newtheorem{proposition}[theorem]{Proposition}
 \newtheorem{statement}[theorem]{Statement}
 \newtheorem{conjecture}[theorem]{Conjecture}
 %\theoremstyle{definition}
 %\newtheorem{definition}[theorem]{Definition}
 \theoremstyle{remark}
 \newtheorem{remark}[theorem]{Remark}
 \numberwithin{equation}{section}



% Set math commands
\newcommand{\conv}    {{\rm{conv}}}
\newcommand{\cl}      {{\rm{cl}}}
\newcommand{\deq}     {\,{\buildrel d \over =}\,}
\newcommand{\dom}      {{\rm{dom}}}
\newcommand{\epi}      {{\rm{epi}}}
\newcommand{\inv}     {^{-1}}
\newcommand{\interior}{{\rm{int}}}
\newcommand{\nab}  [2]{\nabla_{\!#1}^{#2}}
\newcommand{\prob} [1]{P\{#1\}}
\newcommand{\pto}     {\,{\buildrel p \over \to}\,}
\newcommand{\R}    [1]{\mathbb{R}^{#1}}
\newcommand{\st}      {{\rm{s.t.}}}
%\newcommand{\thalf}[1]{\textstyle{\frac{#1}{2}}}
\newcommand{\half}{\frac{1}{2}}
\newcommand{\thalf}{\tfrac{1}{2}}
\newcommand{\tr}      {\text{tr}}
\newcommand{\minimize}{\text{minimize}}
\newcommand{\subjectto}{\text{subject to}}
\newcommand{\dist}{\text{dist}}
\newcommand{\eps}{\epsilon}
% Set redefined commands
\renewcommand{\(}{\left(}
\renewcommand{\)}{\right)}
\newcommand{\sign}{{\rm{sign}}}
\renewcommand{\mid}{{\rm{mid}}}

\newcommand{\norm}[1]{\left\Vert #1\right\Vert}
\newcommand{\onorm}[1]{\norm{#1}_1}
\newcommand{\tnorm}[1]{\norm{#1}_2}
\newcommand{\inorm}[1]{\norm{#1}_\infty}
\newcommand{\tnorms}[1]{\norm{#1}_2^2}
\newcommand{\set}[2]{\left\{#1\,\left|\, #2\right.\right\}}
\newcommand{\ncone}[2]{N\left(#1\, |\, #2\right)}
\newcommand{\tdist}[2]{\mbox{dist}_2\left(#1\,\left|\, #2\right.\right)}
\newcommand{\odist}[2]{\mbox{dist}_1\left(#1\,\left|\, #2\right.\right)}


% Set software
\newcommand{\ampl}   {{\tt AMPL}}
\newcommand{\ipopt}  {{\tt IPOPT}}
\newcommand{\knitro} {{\tt KNITRO}}
\newcommand{\matlab} {{\tt MATLAB}}
\newcommand{\pardiso}{{\tt PARDISO}}
\newcommand{\filter}{{\tt FILTER}}
\newcommand{\lancelot}{{\tt LANCELOT}}
\newcommand{\loqo}{{\tt LOQO}}
\newcommand{\minos}{{\tt MINOS}}
\newcommand{\snopt}{{\tt SNOPT}}
\newcommand{\dir}{{\tt KNITRO-DIRECT}}
\newcommand{\cg}{{\tt KNITRO-CG}}
\newcommand{\act}{{\tt KNITRO-ACTIVE}}
\newcommand{\squid}{{\tt SQuID}}
\newcommand{\mosek}{{\tt MOSEK}}
\newcommand{\slqp}{S$\ell_1$QP }


% Set begin...end commands
\newcommand{\back}{\begin{acknowledgements}} \newcommand{\eack}{\end{acknowledgements}}
\newcommand{\balg}{\begin{algorithm}}    \newcommand{\ealg}{\end{algorithm}}
\newcommand{\balc}{\begin{algorithmic}}  \newcommand{\ealc}{\end{algorithmic}}
\newcommand{\bali}{\begin{aligned}}      \newcommand{\eali}{\end{aligned}}
\newcommand{\barr}{\begin{array}}        \newcommand{\earr}{\end{array}}
\newcommand{\bass}{\begin{assumption}}   \newcommand{\eass}{\end{assumption}}
\newcommand{\bbma}{\begin{bmatrix}}      \newcommand{\ebma}{\end{bmatrix}}
\newcommand{\bcas}{\begin{cases}}        \newcommand{\ecas}{\end{cases}}
\newcommand{\bcen}{\begin{center}}       \newcommand{\ecen}{\end{center}}
\newcommand{\bcol}{\begin{column}}       \newcommand{\ecol}{\end{column}}
\newcommand{\bcos}{\begin{columns}}      \newcommand{\ecos}{\end{columns}}
\newcommand{\bcon}{\begin{condition}}    \newcommand{\econ}{\end{condition}}
\newcommand{\bcor}{\begin{corollary}}    \newcommand{\ecor}{\end{corollary}}
\newcommand{\bdfn}{\begin{definition}}   \newcommand{\edfn}{\end{definition}}
\newcommand{\benu}{\begin{enumerate}}    \newcommand{\eenu}{\end{enumerate}}
\newcommand{\bequ}{\begin{equation}}     \newcommand{\eequ}{\end{equation}}
\newcommand{\benn}{\begin{equation*}}    \newcommand{\eenn}{\end{equation*}}
\newcommand{\bexa}{\begin{example}}      \newcommand{\eexa}{\end{example}}
\newcommand{\bfig}{\begin{figure}}       \newcommand{\efig}{\end{figure}}
\newcommand{\bfra}{\begin{frame}}        \newcommand{\efra}{\end{frame}}
\newcommand{\bite}{\begin{itemize}}      \newcommand{\eite}{\end{itemize}}
\newcommand{\blem}{\begin{lemma}}        \newcommand{\elem}{\end{lemma}}
\newcommand{\bmat}{\begin{matrix}}       \newcommand{\emat}{\end{matrix}}
\newcommand{\bpma}{\begin{pmatrix}}      \newcommand{\epma}{\end{pmatrix}}
\newcommand{\bpic}{\begin{picture}}      \newcommand{\epic}{\end{picture}}
\newcommand{\bpro}{\begin{proof}}        \newcommand{\epro}{\end{proof}}
\newcommand{\bprp}{\begin{proposition}}  \newcommand{\eprp}{\end{proposition}}
\newcommand{\brem}{\begin{remark}}       \newcommand{\erem}{\end{remark}}
\newcommand{\bsub}{\begin{subequations}} \newcommand{\esub}{\end{subequations}}
\newcommand{\btab}{\begin{table}}        \newcommand{\etab}{\end{table}}
\newcommand{\btar}{\begin{tabular}}      \newcommand{\etar}{\end{tabular}}
\newcommand{\bthe}{\begin{theorem}}      \newcommand{\ethe}{\end{theorem}}
\newcommand{\bvma}{\begin{vmatrix}}      \newcommand{\evma}{\end{vmatrix}}
\newcommand{\lsc}{\emph{lsc}\ }
\newcommand{\Min}{\text{minimize}}
\newcommand{\ST}{\text{subject to}}
\newcommand{\beq}{\begin{equation}}
\newcommand{\eeq}{\end{equation}}
\newcommand{\bequation}{\begin{equation}}
\newcommand{\eequation}{\end{equation}}
\newcommand{\bproof}{\begin{proof}}
\newcommand{\eproof}{\end{proof}}
\newcommand{\benumerate}{\begin{enumerate}}   \newcommand{\eenumerate}{\end{enumerate}}
\newcommand{\bitem}{\begin{itemize}}
\newcommand{\eitem}{\end{itemize}}
\newcommand{\bassumption}{\begin{assumption}}
\newcommand{\eassumption}{\end{assumption}}
\newcommand{\bprop}{\begin{proposition}}
\newcommand{\eprop}{\end{proposition}}
\newcommand{\bal}{\begin{aligned}}
\newcommand{\eal}{\end{aligned}}
\newcommand{\baligned}{\begin{aligned}}
\newcommand{\ealigned}{\end{aligned}}
\newcommand{\bseq}{\begin{subequations}}
\newcommand{\eseq}{\end{subequations}}
\newcommand{\bsubequations}{\begin{subequations}}
\newcommand{\esubequations}{\end{subequations}}
\newcommand{\bcases}{\begin{cases}}
\newcommand{\ecases}{\end{cases}}
\newcommand{\bbmatrix}     {\begin{bmatrix}}
\newcommand{\ebmatrix}     {\end{bmatrix}}
\newcommand{\btheorem}{\begin{theorem}}
\newcommand{\etheorem}{\end{theorem}}





% Convex analysis
\newcommand{\indicator}[2]{\delta\left(#1\,\left|\, #2\right.\right)}

% bar and tilde variables
\newcommand{\ol}{\overline}
\newcommand{\wh}{\widehat}
\newcommand{\wt}{\widetilde}
\newcommand{\Lim}{\lim\limits}
\newcommand{\tand}{\text{and}}


\newcommand{\ra}{\rangle}
\newcommand{\la}{\langle}
\newcommand{\shot}{\textbullet}
\newcommand{\diag}{\text{diag}}
\newcommand{\Null}{\text{null}}


% Set text colors
\newcommand{\tblue}  [1]{\textcolor{blue}  {#1}}
\newcommand{\tgreen} [1]{\textcolor{green} {#1}}
\newcommand{\tred}   [1]{\textcolor{red}   {#1}}
\newcommand{\tyellow}[1]{\textcolor{yellow}{#1}}
\definecolor{magenta}{rgb}{1,0,.6}
\newcommand{\tmagenta}[1]{\textcolor{magenta}{#1}}
\newcommand{\green}{\color{green}}
\newcommand{\blue}{\color{blue}}
\newcommand{\red}{\color{red}}
\newcommand{\magenta}{\color{magenta}}




% Set calligraphic letters
\newcommand{\Acal}{{\cal A}}
\newcommand{\Bcal}{{\cal B}}
\newcommand{\Ccal}{{\cal C}}
\newcommand{\Dcal}{{\cal D}}
\newcommand{\Ecal}{{\cal E}}
\newcommand{\Fcal}{{\cal F}}
\newcommand{\Gcal}{{\cal G}}
\newcommand{\Hcal}{{\cal H}}
\newcommand{\Ical}{{\cal I}}
\newcommand{\Jcal}{{\cal J}}
\newcommand{\Kcal}{{\cal K}}
\newcommand{\Lcal}{{\cal L}}
\newcommand{\Mcal}{{\cal M}}
\newcommand{\Ncal}{{\cal N}}
\newcommand{\Ocal}{{\cal O}}
\newcommand{\Pcal}{{\cal P}}
\newcommand{\Qcal}{{\cal Q}}
\newcommand{\Rcal}{{\cal R}}
\newcommand{\Scal}{{\cal S}}
\newcommand{\Tcal}{{\cal T}}
\newcommand{\Ucal}{{\cal U}}
\newcommand{\Vcal}{{\cal V}}
\newcommand{\Wcal}{{\cal W}}
\newcommand{\Xcal}{{\cal X}}
\newcommand{\Ycal}{{\cal Y}}
\newcommand{\Zcal}{{\cal Z}}


% Set blackboard letters
\newcommand{\Ab}{\mathbb{A}}
\newcommand{\Bb}{\mathbb{B}}
\newcommand{\Cb}{\mathbb{C}}
\newcommand{\Db}{\mathbb{D}}
\newcommand{\Eb}{\mathbb{E}}
\newcommand{\Fb}{\mathbb{F}}
\newcommand{\Gb}{\mathbb{G}}
\newcommand{\Hb}{\mathbb{H}}
\newcommand{\Ib}{\mathbb{I}}
\newcommand{\Jb}{\mathbb{J}}
\newcommand{\Kb}{\mathbb{K}}
\newcommand{\Lb}{\mathbb{L}}
\newcommand{\Mb}{\mathbb{M}}
\newcommand{\Nb}{\mathbb{N}}
\newcommand{\Ob}{\mathbb{O}}
\newcommand{\Pb}{\mathbb{P}}
\newcommand{\Qb}{\mathbb{Q}}
\newcommand{\Rb}{\mathbb{R}}
\newcommand{\Sb}{\mathbb{S}}
\newcommand{\Tb}{\mathbb{T}}
\newcommand{\Ub}{\mathbb{U}}
\newcommand{\Vb}{\mathbb{V}}
\newcommand{\Wb}{\mathbb{W}}
\newcommand{\Xb}{\mathbb{X}}
\newcommand{\Yb}{\mathbb{Y}}
\newcommand{\Zb}{\mathbb{Z}}

\newcommand{\bR}{\Rb}
\newcommand{\Rn}{\Rb^n}
\newcommand{\Rm}{\Rb^m}
\newcommand{\Rmn}{\Rb^{m\times n}}
\newcommand{\Rnn}{\Rb^{n\times n}}



% Set bold letters
\newcommand{\zero}{\mathbf{0}}
\newcommand{\one}{\mathbf{1}}

\newcommand{\Abf}{\mathbf{A}}
\newcommand{\abf}{\mathbf{a}}
\newcommand{\Bbf}{\mathbf{B}}
\newcommand{\bbf}{\mathbf{b}}
\newcommand{\Cbf}{\mathbf{C}}
\newcommand{\cbf}{\mathbf{c}}
\newcommand{\Dbf}{\mathbf{D}}
\newcommand{\dbf}{\mathbf{d}}
\newcommand{\Ebf}{\mathbf{E}}
\newcommand{\ebf}{\mathbf{e}}
\newcommand{\Fbf}{\mathbf{F}}
\newcommand{\fbf}{\mathbf{f}}
\newcommand{\Gbf}{\mathbf{G}}
\newcommand{\gbf}{\mathbf{g}}
\newcommand{\Hbf}{\mathbf{H}}
\newcommand{\hbf}{\mathbf{h}}
\newcommand{\Ibf}{\mathbf{I}}
\newcommand{\ibf}{\mathbf{i}}
\newcommand{\Jbf}{\mathbf{J}}
\newcommand{\jbf}{\mathbf{j}}
\newcommand{\Kbf}{\mathbf{K}}
\newcommand{\kbf}{\mathbf{k}}
\newcommand{\Lbf}{\mathbf{L}}
\newcommand{\lbf}{\mathbf{l}}
\newcommand{\Mbf}{\mathbf{M}}
\newcommand{\mbf}{\mathbf{m}}
\newcommand{\Nbf}{\mathbf{N}}
\newcommand{\nbf}{\mathbf{n}}
\newcommand{\Obf}{\mathbf{O}}
\newcommand{\obf}{\mathbf{o}}
\newcommand{\Pbf}{\mathbf{P}}
\newcommand{\pbf}{\mathbf{p}}
\newcommand{\Qbf}{\mathbf{Q}}
\newcommand{\qbf}{\mathbf{q}}
\newcommand{\Rbf}{\mathbf{R}}
\newcommand{\rbf}{\mathbf{r}}
\newcommand{\Sbf}{\mathbf{S}}
\newcommand{\sbf}{\mathbf{s}}
\newcommand{\Tbf}{\mathbf{T}}
\newcommand{\tbf}{\mathbf{t}}
\newcommand{\Ubf}{\mathbf{U}}
\newcommand{\ubf}{\mathbf{u}}
\newcommand{\Vbf}{\mathbf{V}}
\newcommand{\vbf}{\mathbf{v}}
\newcommand{\Wbf}{\mathbf{W}}
\newcommand{\wbf}{\mathbf{w}}
\newcommand{\Xbf}{\mathbf{X}}
\newcommand{\xbf}{\mathbf{x}}
\newcommand{\Ybf}{\mathbf{Y}}
\newcommand{\ybf}{\mathbf{y}}
\newcommand{\Zbf}{\mathbf{Z}}
\newcommand{\zbf}{\mathbf{z}} 



 \newcommand{\sigmabf}{\boldsymbol{\sigma}} 
 \newcommand{\thetabf}{\boldsymbol{\theta}} 










%
%
%\newcommand{\bH}{\mathbf{H}}
%\newcommand{\bS}{\mathbf{S}}
%
%\newcommand{\bg}{\mathbf{g}}
%\newcommand{\btheta}{\mathbf{\theta}}
%\newcommand{\bt}{\mathbf{t}}
%\newcommand{\bX}{\mathbf{X}}
%\newcommand{\bx}{\mathbf{x}}
%\newcommand{\bc}{\mathbf{c}}
%\newcommand{\bp}{\mathbf{p}}
%\newcommand{\bA}{\mathbf{A}}
%\newcommand{\bQ}{\mathbf{Q}}
%\newcommand{\bw}{\mathbf{w}}
%\newcommand{\by}{\mathbf{y}}
%\newcommand{\Ebb}{\mathbb{E}}
%\newcommand{\Rbb}{\mathbb{R}}
%\newcommand{\Pbb}{\mathbb{P}}
%\newcommand{\bz}{\mathbf{z}}
%\newcommand{\bB}{\mathbf{B}}
%\newcommand{\bI}{\mathbf{I}}
%\newcommand{\bW}{\mathbf{W}}
%\newcommand{\bu}{\mathbf{u}}
% Set numbering

\numberwithin{algorithm}{section}
\numberwithin{equation}{section}
\numberwithin{figure}{section}
\numberwithin{table}{section}
\numberwithin{theorem}{section}

% Set language
\usepackage[english]{babel}




%% Set presentation mode
%\mode<presentation>
%{
%  % Set theme
% 
% 
% 
%\usetheme{Madrid}
%%\usecolortheme{beaver}
%%% AnnArbor Antibes Bergen Berkeley beaver Berlin Boadilla boxes CambridgeUS Copenhagen Darmstadt default Dresden
%%% Frankfurt Goettingen Hannover Ilmenau JuanLesPins Luebeck Madrid Malmoe Marburg Montpellier PaloAlto
%%% Pittsburgh Rochester Singapore Szeged Warsaw
%%% To see the combination matrix, go to: http://www.hartwork.org/beamer-theme-matrix/
%  % Set transparent mode
%  \setbeamercovered{transparent}
%}






% Set table of contents mode
\AtBeginSection[]
{
  % Set table of contents slide
  \begin{frame}<beamer>
    \frametitle{Outline}
    \tableofcontents[currentsection]
  \end{frame}
}









% Remove navigation
\setbeamertemplate{navigation symbols}{} 

\setbeamertemplate{footline}[frame number]


\newcommand{\bcls}{\begin{columns}}
\newcommand{\bcl}[1]{\begin{column}{#1\textwidth}}
\newcommand{\ecls}{\end{columns}}
\newcommand{\ecl}{\end{column}}
\newcommand{\blt}{$\bullet$}
\newcommand{\cfig}[2]{\centerline{\includegraphics[scale=#1]{#2}}}
\newcommand{\fig}[2]{\includegraphics[scale=#1]{#2}}

\newcommand{\bicl}[2]{\begin{columns}\begin{column}{0.5\textwidth} #1 \end{column}\begin{column}{0.5\textwidth} #2 \end{column}\end{columns}}


 \usefonttheme[onlymath]{serif}
 \expandafter\def\expandafter\insertshorttitle\expandafter{%  
    \insertshorttitle\hfill%  
    \insertframenumber\,/\,\inserttotalframenumber}  
%% Include packages
\usepackage{ marginnote}
\usepackage[normalem]{ulem}


\usepackage{amsmath,amsfonts,amssymb,latexsym,amsthm,
                         graphicx,graphics,float,subfigure,epsf,enumerate,hyperref,algorithm,algorithmic,multirow,framed}
  
% THEOREM Environments ---------------------------------------------------
%\newtheorem{theorem}{Theorem}[section]
% \newtheorem{corollary}[theorem]{Corollary}
% \newtheorem{lemma}[theorem]{Lemma}
  \newtheorem{assumption}[theorem]{Assumption}
 \newtheorem{proposition}[theorem]{Proposition}
 \newtheorem{statement}[theorem]{Statement}
 \newtheorem{conjecture}[theorem]{Conjecture}
 %\theoremstyle{definition}
 %\newtheorem{definition}[theorem]{Definition}
 \theoremstyle{remark}
 \newtheorem{remark}[theorem]{Remark}
 \numberwithin{equation}{section}



% Set math commands
\newcommand{\conv}    {{\rm{conv}}}
\newcommand{\cl}      {{\rm{cl}}}
\newcommand{\deq}     {\,{\buildrel d \over =}\,}
\newcommand{\dom}      {{\rm{dom}}}
\newcommand{\epi}      {{\rm{epi}}}
\newcommand{\inv}     {^{-1}}
\newcommand{\interior}{{\rm{int}}}
\newcommand{\nab}  [2]{\nabla_{\!#1}^{#2}}
\newcommand{\prob} [1]{P\{#1\}}
\newcommand{\pto}     {\,{\buildrel p \over \to}\,}
\newcommand{\R}    [1]{\mathbb{R}^{#1}}
\newcommand{\st}      {{\rm{s.t.}}}
%\newcommand{\thalf}[1]{\textstyle{\frac{#1}{2}}}
\newcommand{\half}{\frac{1}{2}}
\newcommand{\thalf}{\tfrac{1}{2}}
\newcommand{\tr}      {\text{tr}}
\newcommand{\minimize}{\text{minimize}}
\newcommand{\subjectto}{\text{subject to}}
\newcommand{\dist}{\text{dist}}
\newcommand{\eps}{\epsilon}
% Set redefined commands
\renewcommand{\(}{\left(}
\renewcommand{\)}{\right)}
\newcommand{\sign}{{\rm{sign}}}
\renewcommand{\mid}{{\rm{mid}}}

\newcommand{\norm}[1]{\left\Vert #1\right\Vert}
\newcommand{\onorm}[1]{\norm{#1}_1}
\newcommand{\tnorm}[1]{\norm{#1}_2}
\newcommand{\inorm}[1]{\norm{#1}_\infty}
\newcommand{\tnorms}[1]{\norm{#1}_2^2}
\newcommand{\set}[2]{\left\{#1\,\left|\, #2\right.\right\}}
\newcommand{\ncone}[2]{N\left(#1\, |\, #2\right)}
\newcommand{\tdist}[2]{\mbox{dist}_2\left(#1\,\left|\, #2\right.\right)}
\newcommand{\odist}[2]{\mbox{dist}_1\left(#1\,\left|\, #2\right.\right)}


% Set software
\newcommand{\ampl}   {{\tt AMPL}}
\newcommand{\ipopt}  {{\tt IPOPT}}
\newcommand{\knitro} {{\tt KNITRO}}
\newcommand{\matlab} {{\tt MATLAB}}
\newcommand{\pardiso}{{\tt PARDISO}}
\newcommand{\filter}{{\tt FILTER}}
\newcommand{\lancelot}{{\tt LANCELOT}}
\newcommand{\loqo}{{\tt LOQO}}
\newcommand{\minos}{{\tt MINOS}}
\newcommand{\snopt}{{\tt SNOPT}}
\newcommand{\dir}{{\tt KNITRO-DIRECT}}
\newcommand{\cg}{{\tt KNITRO-CG}}
\newcommand{\act}{{\tt KNITRO-ACTIVE}}
\newcommand{\squid}{{\tt SQuID}}
\newcommand{\mosek}{{\tt MOSEK}}
\newcommand{\slqp}{S$\ell_1$QP }


% Set begin...end commands
\newcommand{\back}{\begin{acknowledgements}} \newcommand{\eack}{\end{acknowledgements}}
\newcommand{\balg}{\begin{algorithm}}    \newcommand{\ealg}{\end{algorithm}}
\newcommand{\balc}{\begin{algorithmic}}  \newcommand{\ealc}{\end{algorithmic}}
\newcommand{\bali}{\begin{aligned}}      \newcommand{\eali}{\end{aligned}}
\newcommand{\barr}{\begin{array}}        \newcommand{\earr}{\end{array}}
\newcommand{\bass}{\begin{assumption}}   \newcommand{\eass}{\end{assumption}}
\newcommand{\bbma}{\begin{bmatrix}}      \newcommand{\ebma}{\end{bmatrix}}
\newcommand{\bcas}{\begin{cases}}        \newcommand{\ecas}{\end{cases}}
\newcommand{\bcen}{\begin{center}}       \newcommand{\ecen}{\end{center}}
\newcommand{\bcol}{\begin{column}}       \newcommand{\ecol}{\end{column}}
\newcommand{\bcos}{\begin{columns}}      \newcommand{\ecos}{\end{columns}}
\newcommand{\bcon}{\begin{condition}}    \newcommand{\econ}{\end{condition}}
\newcommand{\bcor}{\begin{corollary}}    \newcommand{\ecor}{\end{corollary}}
\newcommand{\bdfn}{\begin{definition}}   \newcommand{\edfn}{\end{definition}}
\newcommand{\benu}{\begin{enumerate}}    \newcommand{\eenu}{\end{enumerate}}
\newcommand{\bequ}{\begin{equation}}     \newcommand{\eequ}{\end{equation}}
\newcommand{\benn}{\begin{equation*}}    \newcommand{\eenn}{\end{equation*}}
\newcommand{\bexa}{\begin{example}}      \newcommand{\eexa}{\end{example}}
\newcommand{\bfig}{\begin{figure}}       \newcommand{\efig}{\end{figure}}
\newcommand{\bfra}{\begin{frame}}        \newcommand{\efra}{\end{frame}}
\newcommand{\bite}{\begin{itemize}}      \newcommand{\eite}{\end{itemize}}
\newcommand{\blem}{\begin{lemma}}        \newcommand{\elem}{\end{lemma}}
\newcommand{\bmat}{\begin{matrix}}       \newcommand{\emat}{\end{matrix}}
\newcommand{\bpma}{\begin{pmatrix}}      \newcommand{\epma}{\end{pmatrix}}
\newcommand{\bpic}{\begin{picture}}      \newcommand{\epic}{\end{picture}}
\newcommand{\bpro}{\begin{proof}}        \newcommand{\epro}{\end{proof}}
\newcommand{\bprp}{\begin{proposition}}  \newcommand{\eprp}{\end{proposition}}
\newcommand{\brem}{\begin{remark}}       \newcommand{\erem}{\end{remark}}
\newcommand{\bsub}{\begin{subequations}} \newcommand{\esub}{\end{subequations}}
\newcommand{\btab}{\begin{table}}        \newcommand{\etab}{\end{table}}
\newcommand{\btar}{\begin{tabular}}      \newcommand{\etar}{\end{tabular}}
\newcommand{\bthe}{\begin{theorem}}      \newcommand{\ethe}{\end{theorem}}
\newcommand{\bvma}{\begin{vmatrix}}      \newcommand{\evma}{\end{vmatrix}}
\newcommand{\lsc}{\emph{lsc}\ }
\newcommand{\Min}{\text{minimize}}
\newcommand{\ST}{\text{subject to}}
\newcommand{\beq}{\begin{equation}}
\newcommand{\eeq}{\end{equation}}
\newcommand{\bequation}{\begin{equation}}
\newcommand{\eequation}{\end{equation}}
\newcommand{\bproof}{\begin{proof}}
\newcommand{\eproof}{\end{proof}}
\newcommand{\benumerate}{\begin{enumerate}}   \newcommand{\eenumerate}{\end{enumerate}}
\newcommand{\bitem}{\begin{itemize}}
\newcommand{\eitem}{\end{itemize}}
\newcommand{\bassumption}{\begin{assumption}}
\newcommand{\eassumption}{\end{assumption}}
\newcommand{\bprop}{\begin{proposition}}
\newcommand{\eprop}{\end{proposition}}
\newcommand{\bal}{\begin{aligned}}
\newcommand{\eal}{\end{aligned}}
\newcommand{\baligned}{\begin{aligned}}
\newcommand{\ealigned}{\end{aligned}}
\newcommand{\bseq}{\begin{subequations}}
\newcommand{\eseq}{\end{subequations}}
\newcommand{\bsubequations}{\begin{subequations}}
\newcommand{\esubequations}{\end{subequations}}
\newcommand{\bcases}{\begin{cases}}
\newcommand{\ecases}{\end{cases}}
\newcommand{\bbmatrix}     {\begin{bmatrix}}
\newcommand{\ebmatrix}     {\end{bmatrix}}
\newcommand{\btheorem}{\begin{theorem}}
\newcommand{\etheorem}{\end{theorem}}





% Convex analysis
\newcommand{\indicator}[2]{\delta\left(#1\,\left|\, #2\right.\right)}

% bar and tilde variables
\newcommand{\ol}{\overline}
\newcommand{\wh}{\widehat}
\newcommand{\wt}{\widetilde}
\newcommand{\Lim}{\lim\limits}
\newcommand{\tand}{\text{and}}


\newcommand{\ra}{\rangle}
\newcommand{\la}{\langle}
\newcommand{\shot}{\textbullet}
\newcommand{\diag}{\text{diag}}
\newcommand{\Null}{\text{null}}


% Set text colors
\newcommand{\tblue}  [1]{\textcolor{blue}  {#1}}
\newcommand{\tgreen} [1]{\textcolor{green} {#1}}
\newcommand{\tred}   [1]{\textcolor{red}   {#1}}
\newcommand{\tyellow}[1]{\textcolor{yellow}{#1}}
\definecolor{magenta}{rgb}{1,0,.6}
\newcommand{\tmagenta}[1]{\textcolor{magenta}{#1}}
\newcommand{\green}{\color{green}}
\newcommand{\blue}{\color{blue}}
\newcommand{\red}{\color{red}}
\newcommand{\magenta}{\color{magenta}}




% Set calligraphic letters
\newcommand{\Acal}{{\cal A}}
\newcommand{\Bcal}{{\cal B}}
\newcommand{\Ccal}{{\cal C}}
\newcommand{\Dcal}{{\cal D}}
\newcommand{\Ecal}{{\cal E}}
\newcommand{\Fcal}{{\cal F}}
\newcommand{\Gcal}{{\cal G}}
\newcommand{\Hcal}{{\cal H}}
\newcommand{\Ical}{{\cal I}}
\newcommand{\Jcal}{{\cal J}}
\newcommand{\Kcal}{{\cal K}}
\newcommand{\Lcal}{{\cal L}}
\newcommand{\Mcal}{{\cal M}}
\newcommand{\Ncal}{{\cal N}}
\newcommand{\Ocal}{{\cal O}}
\newcommand{\Pcal}{{\cal P}}
\newcommand{\Qcal}{{\cal Q}}
\newcommand{\Rcal}{{\cal R}}
\newcommand{\Scal}{{\cal S}}
\newcommand{\Tcal}{{\cal T}}
\newcommand{\Ucal}{{\cal U}}
\newcommand{\Vcal}{{\cal V}}
\newcommand{\Wcal}{{\cal W}}
\newcommand{\Xcal}{{\cal X}}
\newcommand{\Ycal}{{\cal Y}}
\newcommand{\Zcal}{{\cal Z}}


% Set blackboard letters
\newcommand{\Ab}{\mathbb{A}}
\newcommand{\Bb}{\mathbb{B}}
\newcommand{\Cb}{\mathbb{C}}
\newcommand{\Db}{\mathbb{D}}
\newcommand{\Eb}{\mathbb{E}}
\newcommand{\Fb}{\mathbb{F}}
\newcommand{\Gb}{\mathbb{G}}
\newcommand{\Hb}{\mathbb{H}}
\newcommand{\Ib}{\mathbb{I}}
\newcommand{\Jb}{\mathbb{J}}
\newcommand{\Kb}{\mathbb{K}}
\newcommand{\Lb}{\mathbb{L}}
\newcommand{\Mb}{\mathbb{M}}
\newcommand{\Nb}{\mathbb{N}}
\newcommand{\Ob}{\mathbb{O}}
\newcommand{\Pb}{\mathbb{P}}
\newcommand{\Qb}{\mathbb{Q}}
\newcommand{\Rb}{\mathbb{R}}
\newcommand{\Sb}{\mathbb{S}}
\newcommand{\Tb}{\mathbb{T}}
\newcommand{\Ub}{\mathbb{U}}
\newcommand{\Vb}{\mathbb{V}}
\newcommand{\Wb}{\mathbb{W}}
\newcommand{\Xb}{\mathbb{X}}
\newcommand{\Yb}{\mathbb{Y}}
\newcommand{\Zb}{\mathbb{Z}}

\newcommand{\bR}{\Rb}
\newcommand{\Rn}{\Rb^n}
\newcommand{\Rm}{\Rb^m}
\newcommand{\Rmn}{\Rb^{m\times n}}
\newcommand{\Rnn}{\Rb^{n\times n}}



% Set bold letters
\newcommand{\zero}{\mathbf{0}}
\newcommand{\one}{\mathbf{1}}

\newcommand{\Abf}{\mathbf{A}}
\newcommand{\abf}{\mathbf{a}}
\newcommand{\Bbf}{\mathbf{B}}
\newcommand{\bbf}{\mathbf{b}}
\newcommand{\Cbf}{\mathbf{C}}
\newcommand{\cbf}{\mathbf{c}}
\newcommand{\Dbf}{\mathbf{D}}
\newcommand{\dbf}{\mathbf{d}}
\newcommand{\Ebf}{\mathbf{E}}
\newcommand{\ebf}{\mathbf{e}}
\newcommand{\Fbf}{\mathbf{F}}
\newcommand{\fbf}{\mathbf{f}}
\newcommand{\Gbf}{\mathbf{G}}
\newcommand{\gbf}{\mathbf{g}}
\newcommand{\Hbf}{\mathbf{H}}
\newcommand{\hbf}{\mathbf{h}}
\newcommand{\Ibf}{\mathbf{I}}
\newcommand{\ibf}{\mathbf{i}}
\newcommand{\Jbf}{\mathbf{J}}
\newcommand{\jbf}{\mathbf{j}}
\newcommand{\Kbf}{\mathbf{K}}
\newcommand{\kbf}{\mathbf{k}}
\newcommand{\Lbf}{\mathbf{L}}
\newcommand{\lbf}{\mathbf{l}}
\newcommand{\Mbf}{\mathbf{M}}
\newcommand{\mbf}{\mathbf{m}}
\newcommand{\Nbf}{\mathbf{N}}
\newcommand{\nbf}{\mathbf{n}}
\newcommand{\Obf}{\mathbf{O}}
\newcommand{\obf}{\mathbf{o}}
\newcommand{\Pbf}{\mathbf{P}}
\newcommand{\pbf}{\mathbf{p}}
\newcommand{\Qbf}{\mathbf{Q}}
\newcommand{\qbf}{\mathbf{q}}
\newcommand{\Rbf}{\mathbf{R}}
\newcommand{\rbf}{\mathbf{r}}
\newcommand{\Sbf}{\mathbf{S}}
\newcommand{\sbf}{\mathbf{s}}
\newcommand{\Tbf}{\mathbf{T}}
\newcommand{\tbf}{\mathbf{t}}
\newcommand{\Ubf}{\mathbf{U}}
\newcommand{\ubf}{\mathbf{u}}
\newcommand{\Vbf}{\mathbf{V}}
\newcommand{\vbf}{\mathbf{v}}
\newcommand{\Wbf}{\mathbf{W}}
\newcommand{\wbf}{\mathbf{w}}
\newcommand{\Xbf}{\mathbf{X}}
\newcommand{\xbf}{\mathbf{x}}
\newcommand{\Ybf}{\mathbf{Y}}
\newcommand{\ybf}{\mathbf{y}}
\newcommand{\Zbf}{\mathbf{Z}}
\newcommand{\zbf}{\mathbf{z}} 



 \newcommand{\sigmabf}{\boldsymbol{\sigma}} 
 \newcommand{\thetabf}{\boldsymbol{\theta}} 










%
%
%\newcommand{\bH}{\mathbf{H}}
%\newcommand{\bS}{\mathbf{S}}
%
%\newcommand{\bg}{\mathbf{g}}
%\newcommand{\btheta}{\mathbf{\theta}}
%\newcommand{\bt}{\mathbf{t}}
%\newcommand{\bX}{\mathbf{X}}
%\newcommand{\bx}{\mathbf{x}}
%\newcommand{\bc}{\mathbf{c}}
%\newcommand{\bp}{\mathbf{p}}
%\newcommand{\bA}{\mathbf{A}}
%\newcommand{\bQ}{\mathbf{Q}}
%\newcommand{\bw}{\mathbf{w}}
%\newcommand{\by}{\mathbf{y}}
%\newcommand{\Ebb}{\mathbb{E}}
%\newcommand{\Rbb}{\mathbb{R}}
%\newcommand{\Pbb}{\mathbb{P}}
%\newcommand{\bz}{\mathbf{z}}
%\newcommand{\bB}{\mathbf{B}}
%\newcommand{\bI}{\mathbf{I}}
%\newcommand{\bW}{\mathbf{W}}
%\newcommand{\bu}{\mathbf{u}}
% Set numbering

\numberwithin{algorithm}{section}
\numberwithin{equation}{section}
\numberwithin{figure}{section}
\numberwithin{table}{section}
\numberwithin{theorem}{section}

% Set language
\usepackage[english]{babel}




%% Set presentation mode
%\mode<presentation>
%{
%  % Set theme
% 
% 
% 
%\usetheme{Madrid}
%%\usecolortheme{beaver}
%%% AnnArbor Antibes Bergen Berkeley beaver Berlin Boadilla boxes CambridgeUS Copenhagen Darmstadt default Dresden
%%% Frankfurt Goettingen Hannover Ilmenau JuanLesPins Luebeck Madrid Malmoe Marburg Montpellier PaloAlto
%%% Pittsburgh Rochester Singapore Szeged Warsaw
%%% To see the combination matrix, go to: http://www.hartwork.org/beamer-theme-matrix/
%  % Set transparent mode
%  \setbeamercovered{transparent}
%}






% Set table of contents mode
\AtBeginSection[]
{
  % Set table of contents slide
  \begin{frame}<beamer>
    \frametitle{Outline}
    \tableofcontents[currentsection]
  \end{frame}
}









% Remove navigation
\setbeamertemplate{navigation symbols}{} 

\setbeamertemplate{footline}[frame number]


\newcommand{\bcls}{\begin{columns}}
\newcommand{\bcl}[1]{\begin{column}{#1\textwidth}}
\newcommand{\ecls}{\end{columns}}
\newcommand{\ecl}{\end{column}}
\newcommand{\blt}{$\bullet$}
\newcommand{\cfig}[2]{\centerline{\includegraphics[scale=#1]{#2}}}
\newcommand{\fig}[2]{\includegraphics[scale=#1]{#2}}

\newcommand{\bicl}[2]{\begin{columns}\begin{column}{0.5\textwidth} #1 \end{column}\begin{column}{0.5\textwidth} #2 \end{column}\end{columns}}


 \usefonttheme[onlymath]{serif}
 \expandafter\def\expandafter\insertshorttitle\expandafter{%  
    \insertshorttitle\hfill%  
    \insertframenumber\,/\,\inserttotalframenumber}  
%% Include packages
\usepackage{ marginnote}
\usepackage[normalem]{ulem}


\usepackage{amsmath,amsfonts,amssymb,latexsym,amsthm,
                         graphicx,graphics,float,subfigure,epsf,enumerate,hyperref,algorithm,algorithmic,multirow,framed}
  
% THEOREM Environments ---------------------------------------------------
%\newtheorem{theorem}{Theorem}[section]
% \newtheorem{corollary}[theorem]{Corollary}
% \newtheorem{lemma}[theorem]{Lemma}
  \newtheorem{assumption}[theorem]{Assumption}
 \newtheorem{proposition}[theorem]{Proposition}
 \newtheorem{statement}[theorem]{Statement}
 \newtheorem{conjecture}[theorem]{Conjecture}
 %\theoremstyle{definition}
 %\newtheorem{definition}[theorem]{Definition}
 \theoremstyle{remark}
 \newtheorem{remark}[theorem]{Remark}
 \numberwithin{equation}{section}



% Set math commands
\newcommand{\conv}    {{\rm{conv}}}
\newcommand{\cl}      {{\rm{cl}}}
\newcommand{\deq}     {\,{\buildrel d \over =}\,}
\newcommand{\dom}      {{\rm{dom}}}
\newcommand{\epi}      {{\rm{epi}}}
\newcommand{\inv}     {^{-1}}
\newcommand{\interior}{{\rm{int}}}
\newcommand{\nab}  [2]{\nabla_{\!#1}^{#2}}
\newcommand{\prob} [1]{P\{#1\}}
\newcommand{\pto}     {\,{\buildrel p \over \to}\,}
\newcommand{\R}    [1]{\mathbb{R}^{#1}}
\newcommand{\st}      {{\rm{s.t.}}}
%\newcommand{\thalf}[1]{\textstyle{\frac{#1}{2}}}
\newcommand{\half}{\frac{1}{2}}
\newcommand{\thalf}{\tfrac{1}{2}}
\newcommand{\tr}      {\text{tr}}
\newcommand{\minimize}{\text{minimize}}
\newcommand{\subjectto}{\text{subject to}}
\newcommand{\dist}{\text{dist}}
\newcommand{\eps}{\epsilon}
% Set redefined commands
\renewcommand{\(}{\left(}
\renewcommand{\)}{\right)}
\newcommand{\sign}{{\rm{sign}}}
\renewcommand{\mid}{{\rm{mid}}}

\newcommand{\norm}[1]{\left\Vert #1\right\Vert}
\newcommand{\onorm}[1]{\norm{#1}_1}
\newcommand{\tnorm}[1]{\norm{#1}_2}
\newcommand{\inorm}[1]{\norm{#1}_\infty}
\newcommand{\tnorms}[1]{\norm{#1}_2^2}
\newcommand{\set}[2]{\left\{#1\,\left|\, #2\right.\right\}}
\newcommand{\ncone}[2]{N\left(#1\, |\, #2\right)}
\newcommand{\tdist}[2]{\mbox{dist}_2\left(#1\,\left|\, #2\right.\right)}
\newcommand{\odist}[2]{\mbox{dist}_1\left(#1\,\left|\, #2\right.\right)}


% Set software
\newcommand{\ampl}   {{\tt AMPL}}
\newcommand{\ipopt}  {{\tt IPOPT}}
\newcommand{\knitro} {{\tt KNITRO}}
\newcommand{\matlab} {{\tt MATLAB}}
\newcommand{\pardiso}{{\tt PARDISO}}
\newcommand{\filter}{{\tt FILTER}}
\newcommand{\lancelot}{{\tt LANCELOT}}
\newcommand{\loqo}{{\tt LOQO}}
\newcommand{\minos}{{\tt MINOS}}
\newcommand{\snopt}{{\tt SNOPT}}
\newcommand{\dir}{{\tt KNITRO-DIRECT}}
\newcommand{\cg}{{\tt KNITRO-CG}}
\newcommand{\act}{{\tt KNITRO-ACTIVE}}
\newcommand{\squid}{{\tt SQuID}}
\newcommand{\mosek}{{\tt MOSEK}}
\newcommand{\slqp}{S$\ell_1$QP }


% Set begin...end commands
\newcommand{\back}{\begin{acknowledgements}} \newcommand{\eack}{\end{acknowledgements}}
\newcommand{\balg}{\begin{algorithm}}    \newcommand{\ealg}{\end{algorithm}}
\newcommand{\balc}{\begin{algorithmic}}  \newcommand{\ealc}{\end{algorithmic}}
\newcommand{\bali}{\begin{aligned}}      \newcommand{\eali}{\end{aligned}}
\newcommand{\barr}{\begin{array}}        \newcommand{\earr}{\end{array}}
\newcommand{\bass}{\begin{assumption}}   \newcommand{\eass}{\end{assumption}}
\newcommand{\bbma}{\begin{bmatrix}}      \newcommand{\ebma}{\end{bmatrix}}
\newcommand{\bcas}{\begin{cases}}        \newcommand{\ecas}{\end{cases}}
\newcommand{\bcen}{\begin{center}}       \newcommand{\ecen}{\end{center}}
\newcommand{\bcol}{\begin{column}}       \newcommand{\ecol}{\end{column}}
\newcommand{\bcos}{\begin{columns}}      \newcommand{\ecos}{\end{columns}}
\newcommand{\bcon}{\begin{condition}}    \newcommand{\econ}{\end{condition}}
\newcommand{\bcor}{\begin{corollary}}    \newcommand{\ecor}{\end{corollary}}
\newcommand{\bdfn}{\begin{definition}}   \newcommand{\edfn}{\end{definition}}
\newcommand{\benu}{\begin{enumerate}}    \newcommand{\eenu}{\end{enumerate}}
\newcommand{\bequ}{\begin{equation}}     \newcommand{\eequ}{\end{equation}}
\newcommand{\benn}{\begin{equation*}}    \newcommand{\eenn}{\end{equation*}}
\newcommand{\bexa}{\begin{example}}      \newcommand{\eexa}{\end{example}}
\newcommand{\bfig}{\begin{figure}}       \newcommand{\efig}{\end{figure}}
\newcommand{\bfra}{\begin{frame}}        \newcommand{\efra}{\end{frame}}
\newcommand{\bite}{\begin{itemize}}      \newcommand{\eite}{\end{itemize}}
\newcommand{\blem}{\begin{lemma}}        \newcommand{\elem}{\end{lemma}}
\newcommand{\bmat}{\begin{matrix}}       \newcommand{\emat}{\end{matrix}}
\newcommand{\bpma}{\begin{pmatrix}}      \newcommand{\epma}{\end{pmatrix}}
\newcommand{\bpic}{\begin{picture}}      \newcommand{\epic}{\end{picture}}
\newcommand{\bpro}{\begin{proof}}        \newcommand{\epro}{\end{proof}}
\newcommand{\bprp}{\begin{proposition}}  \newcommand{\eprp}{\end{proposition}}
\newcommand{\brem}{\begin{remark}}       \newcommand{\erem}{\end{remark}}
\newcommand{\bsub}{\begin{subequations}} \newcommand{\esub}{\end{subequations}}
\newcommand{\btab}{\begin{table}}        \newcommand{\etab}{\end{table}}
\newcommand{\btar}{\begin{tabular}}      \newcommand{\etar}{\end{tabular}}
\newcommand{\bthe}{\begin{theorem}}      \newcommand{\ethe}{\end{theorem}}
\newcommand{\bvma}{\begin{vmatrix}}      \newcommand{\evma}{\end{vmatrix}}
\newcommand{\lsc}{\emph{lsc}\ }
\newcommand{\Min}{\text{minimize}}
\newcommand{\ST}{\text{subject to}}
\newcommand{\beq}{\begin{equation}}
\newcommand{\eeq}{\end{equation}}
\newcommand{\bequation}{\begin{equation}}
\newcommand{\eequation}{\end{equation}}
\newcommand{\bproof}{\begin{proof}}
\newcommand{\eproof}{\end{proof}}
\newcommand{\benumerate}{\begin{enumerate}}   \newcommand{\eenumerate}{\end{enumerate}}
\newcommand{\bitem}{\begin{itemize}}
\newcommand{\eitem}{\end{itemize}}
\newcommand{\bassumption}{\begin{assumption}}
\newcommand{\eassumption}{\end{assumption}}
\newcommand{\bprop}{\begin{proposition}}
\newcommand{\eprop}{\end{proposition}}
\newcommand{\bal}{\begin{aligned}}
\newcommand{\eal}{\end{aligned}}
\newcommand{\baligned}{\begin{aligned}}
\newcommand{\ealigned}{\end{aligned}}
\newcommand{\bseq}{\begin{subequations}}
\newcommand{\eseq}{\end{subequations}}
\newcommand{\bsubequations}{\begin{subequations}}
\newcommand{\esubequations}{\end{subequations}}
\newcommand{\bcases}{\begin{cases}}
\newcommand{\ecases}{\end{cases}}
\newcommand{\bbmatrix}     {\begin{bmatrix}}
\newcommand{\ebmatrix}     {\end{bmatrix}}
\newcommand{\btheorem}{\begin{theorem}}
\newcommand{\etheorem}{\end{theorem}}





% Convex analysis
\newcommand{\indicator}[2]{\delta\left(#1\,\left|\, #2\right.\right)}

% bar and tilde variables
\newcommand{\ol}{\overline}
\newcommand{\wh}{\widehat}
\newcommand{\wt}{\widetilde}
\newcommand{\Lim}{\lim\limits}
\newcommand{\tand}{\text{and}}


\newcommand{\ra}{\rangle}
\newcommand{\la}{\langle}
\newcommand{\shot}{\textbullet}
\newcommand{\diag}{\text{diag}}
\newcommand{\Null}{\text{null}}


% Set text colors
\newcommand{\tblue}  [1]{\textcolor{blue}  {#1}}
\newcommand{\tgreen} [1]{\textcolor{green} {#1}}
\newcommand{\tred}   [1]{\textcolor{red}   {#1}}
\newcommand{\tyellow}[1]{\textcolor{yellow}{#1}}
\definecolor{magenta}{rgb}{1,0,.6}
\newcommand{\tmagenta}[1]{\textcolor{magenta}{#1}}
\newcommand{\green}{\color{green}}
\newcommand{\blue}{\color{blue}}
\newcommand{\red}{\color{red}}
\newcommand{\magenta}{\color{magenta}}




% Set calligraphic letters
\newcommand{\Acal}{{\cal A}}
\newcommand{\Bcal}{{\cal B}}
\newcommand{\Ccal}{{\cal C}}
\newcommand{\Dcal}{{\cal D}}
\newcommand{\Ecal}{{\cal E}}
\newcommand{\Fcal}{{\cal F}}
\newcommand{\Gcal}{{\cal G}}
\newcommand{\Hcal}{{\cal H}}
\newcommand{\Ical}{{\cal I}}
\newcommand{\Jcal}{{\cal J}}
\newcommand{\Kcal}{{\cal K}}
\newcommand{\Lcal}{{\cal L}}
\newcommand{\Mcal}{{\cal M}}
\newcommand{\Ncal}{{\cal N}}
\newcommand{\Ocal}{{\cal O}}
\newcommand{\Pcal}{{\cal P}}
\newcommand{\Qcal}{{\cal Q}}
\newcommand{\Rcal}{{\cal R}}
\newcommand{\Scal}{{\cal S}}
\newcommand{\Tcal}{{\cal T}}
\newcommand{\Ucal}{{\cal U}}
\newcommand{\Vcal}{{\cal V}}
\newcommand{\Wcal}{{\cal W}}
\newcommand{\Xcal}{{\cal X}}
\newcommand{\Ycal}{{\cal Y}}
\newcommand{\Zcal}{{\cal Z}}


% Set blackboard letters
\newcommand{\Ab}{\mathbb{A}}
\newcommand{\Bb}{\mathbb{B}}
\newcommand{\Cb}{\mathbb{C}}
\newcommand{\Db}{\mathbb{D}}
\newcommand{\Eb}{\mathbb{E}}
\newcommand{\Fb}{\mathbb{F}}
\newcommand{\Gb}{\mathbb{G}}
\newcommand{\Hb}{\mathbb{H}}
\newcommand{\Ib}{\mathbb{I}}
\newcommand{\Jb}{\mathbb{J}}
\newcommand{\Kb}{\mathbb{K}}
\newcommand{\Lb}{\mathbb{L}}
\newcommand{\Mb}{\mathbb{M}}
\newcommand{\Nb}{\mathbb{N}}
\newcommand{\Ob}{\mathbb{O}}
\newcommand{\Pb}{\mathbb{P}}
\newcommand{\Qb}{\mathbb{Q}}
\newcommand{\Rb}{\mathbb{R}}
\newcommand{\Sb}{\mathbb{S}}
\newcommand{\Tb}{\mathbb{T}}
\newcommand{\Ub}{\mathbb{U}}
\newcommand{\Vb}{\mathbb{V}}
\newcommand{\Wb}{\mathbb{W}}
\newcommand{\Xb}{\mathbb{X}}
\newcommand{\Yb}{\mathbb{Y}}
\newcommand{\Zb}{\mathbb{Z}}

\newcommand{\bR}{\Rb}
\newcommand{\Rn}{\Rb^n}
\newcommand{\Rm}{\Rb^m}
\newcommand{\Rmn}{\Rb^{m\times n}}
\newcommand{\Rnn}{\Rb^{n\times n}}



% Set bold letters
\newcommand{\zero}{\mathbf{0}}
\newcommand{\one}{\mathbf{1}}

\newcommand{\Abf}{\mathbf{A}}
\newcommand{\abf}{\mathbf{a}}
\newcommand{\Bbf}{\mathbf{B}}
\newcommand{\bbf}{\mathbf{b}}
\newcommand{\Cbf}{\mathbf{C}}
\newcommand{\cbf}{\mathbf{c}}
\newcommand{\Dbf}{\mathbf{D}}
\newcommand{\dbf}{\mathbf{d}}
\newcommand{\Ebf}{\mathbf{E}}
\newcommand{\ebf}{\mathbf{e}}
\newcommand{\Fbf}{\mathbf{F}}
\newcommand{\fbf}{\mathbf{f}}
\newcommand{\Gbf}{\mathbf{G}}
\newcommand{\gbf}{\mathbf{g}}
\newcommand{\Hbf}{\mathbf{H}}
\newcommand{\hbf}{\mathbf{h}}
\newcommand{\Ibf}{\mathbf{I}}
\newcommand{\ibf}{\mathbf{i}}
\newcommand{\Jbf}{\mathbf{J}}
\newcommand{\jbf}{\mathbf{j}}
\newcommand{\Kbf}{\mathbf{K}}
\newcommand{\kbf}{\mathbf{k}}
\newcommand{\Lbf}{\mathbf{L}}
\newcommand{\lbf}{\mathbf{l}}
\newcommand{\Mbf}{\mathbf{M}}
\newcommand{\mbf}{\mathbf{m}}
\newcommand{\Nbf}{\mathbf{N}}
\newcommand{\nbf}{\mathbf{n}}
\newcommand{\Obf}{\mathbf{O}}
\newcommand{\obf}{\mathbf{o}}
\newcommand{\Pbf}{\mathbf{P}}
\newcommand{\pbf}{\mathbf{p}}
\newcommand{\Qbf}{\mathbf{Q}}
\newcommand{\qbf}{\mathbf{q}}
\newcommand{\Rbf}{\mathbf{R}}
\newcommand{\rbf}{\mathbf{r}}
\newcommand{\Sbf}{\mathbf{S}}
\newcommand{\sbf}{\mathbf{s}}
\newcommand{\Tbf}{\mathbf{T}}
\newcommand{\tbf}{\mathbf{t}}
\newcommand{\Ubf}{\mathbf{U}}
\newcommand{\ubf}{\mathbf{u}}
\newcommand{\Vbf}{\mathbf{V}}
\newcommand{\vbf}{\mathbf{v}}
\newcommand{\Wbf}{\mathbf{W}}
\newcommand{\wbf}{\mathbf{w}}
\newcommand{\Xbf}{\mathbf{X}}
\newcommand{\xbf}{\mathbf{x}}
\newcommand{\Ybf}{\mathbf{Y}}
\newcommand{\ybf}{\mathbf{y}}
\newcommand{\Zbf}{\mathbf{Z}}
\newcommand{\zbf}{\mathbf{z}} 



 \newcommand{\sigmabf}{\boldsymbol{\sigma}} 
 \newcommand{\thetabf}{\boldsymbol{\theta}} 










%
%
%\newcommand{\bH}{\mathbf{H}}
%\newcommand{\bS}{\mathbf{S}}
%
%\newcommand{\bg}{\mathbf{g}}
%\newcommand{\btheta}{\mathbf{\theta}}
%\newcommand{\bt}{\mathbf{t}}
%\newcommand{\bX}{\mathbf{X}}
%\newcommand{\bx}{\mathbf{x}}
%\newcommand{\bc}{\mathbf{c}}
%\newcommand{\bp}{\mathbf{p}}
%\newcommand{\bA}{\mathbf{A}}
%\newcommand{\bQ}{\mathbf{Q}}
%\newcommand{\bw}{\mathbf{w}}
%\newcommand{\by}{\mathbf{y}}
%\newcommand{\Ebb}{\mathbb{E}}
%\newcommand{\Rbb}{\mathbb{R}}
%\newcommand{\Pbb}{\mathbb{P}}
%\newcommand{\bz}{\mathbf{z}}
%\newcommand{\bB}{\mathbf{B}}
%\newcommand{\bI}{\mathbf{I}}
%\newcommand{\bW}{\mathbf{W}}
%\newcommand{\bu}{\mathbf{u}}
% Set numbering

\numberwithin{algorithm}{section}
\numberwithin{equation}{section}
\numberwithin{figure}{section}
\numberwithin{table}{section}
\numberwithin{theorem}{section}

% Set language
\usepackage[english]{babel}




%% Set presentation mode
%\mode<presentation>
%{
%  % Set theme
% 
% 
% 
%\usetheme{Madrid}
%%\usecolortheme{beaver}
%%% AnnArbor Antibes Bergen Berkeley beaver Berlin Boadilla boxes CambridgeUS Copenhagen Darmstadt default Dresden
%%% Frankfurt Goettingen Hannover Ilmenau JuanLesPins Luebeck Madrid Malmoe Marburg Montpellier PaloAlto
%%% Pittsburgh Rochester Singapore Szeged Warsaw
%%% To see the combination matrix, go to: http://www.hartwork.org/beamer-theme-matrix/
%  % Set transparent mode
%  \setbeamercovered{transparent}
%}






% Set table of contents mode
\AtBeginSection[]
{
  % Set table of contents slide
  \begin{frame}<beamer>
    \frametitle{Outline}
    \tableofcontents[currentsection]
  \end{frame}
}









% Remove navigation
\setbeamertemplate{navigation symbols}{} 

\setbeamertemplate{footline}[frame number]


\newcommand{\bcls}{\begin{columns}}
\newcommand{\bcl}[1]{\begin{column}{#1\textwidth}}
\newcommand{\ecls}{\end{columns}}
\newcommand{\ecl}{\end{column}}
\newcommand{\blt}{$\bullet$}
\newcommand{\cfig}[2]{\centerline{\includegraphics[scale=#1]{#2}}}
\newcommand{\fig}[2]{\includegraphics[scale=#1]{#2}}

\newcommand{\bicl}[2]{\begin{columns}\begin{column}{0.5\textwidth} #1 \end{column}\begin{column}{0.5\textwidth} #2 \end{column}\end{columns}}


 \usefonttheme[onlymath]{serif}
 \expandafter\def\expandafter\insertshorttitle\expandafter{%  
    \insertshorttitle\hfill%  
    \insertframenumber\,/\,\inserttotalframenumber}  
%**************

%\newcommand{\norm}[1]{\left\Vert #1\right\Vert}
%\newcommand{\tnorm}[1]{\norm{#1}_2}
%\newcommand{\inorm}[1]{\norm{#1}_\infty}
%\newcommand{\onorm}[1]{\norm{#1}_1}
%\newcommand{\dnorm}[1]{\norm{#1}_*}
%\newcommand{\ip}[2]{\left\langle #1,\, #2\right\rangle}
%\newcommand{\set}[2]{\left\{#1\,\left|\, #2\right.\right\}}
\newcommand{\map}[3]{#1:\, #2\rightarrow #3}
%\newcommand{\mmap}[3]{#1:\, #2\Rightarrow #3}

%\newcommand{\dist}[2]{\mathrm{dist}\left(#1\,\left|\, #2\right.\right)}
%\newcommand{\odist}[2]{\mathrm{dist}_1\left(#1\,\left|\, #2\right.\right)}
%\newcommand{\tdist}[2]{\mathrm{dist}_2\left(#1\,\left|\, #2\right.\right)}
%\newcommand{\sdist}[2]{\mathrm{dist}_2^2\left(#1\,\left|\, #2\right.\right)}

\newcommand{\Jhat}{{\skew6\hat J}}
 
\newcommand{\sd}{{\partial}}


 \usepackage{savesym}
\savesymbol{checkmark}
  \usepackage{pifont,dingbat,halloweenmath}
 
\newcommand{\Pri}{P_{C_i}}
%\newcommand{\pri}{p_i}
%\newcommand{\qri}{q_i}
%\newcommand{\wri}{\tw_i}
%\newcommand{\slqp}{S$\ell_1$QP }
\newcommand{\bx}{{\bar x}}

%**************
% Headings
%**************
% Set title
\title{AN INEXACT FIRST-ORDER METHOD FOR CONSTRAINED NONLINEAR OPTIMIZATION
}

% Set author(s)
\author[Rong, Zhou]{}

% Set institute(s)
\institute[ShanghaiTech]{}

% Set conference
\date{}
%**************

%**************
% Document
%**************
\begin{document}

% Title page
\begin{frame}
  \titlepage
  
  \vspace{-4em}
  \begin{table}
 \begin{tabular}{c@{\hspace{1cm}}c@{\hspace{1cm}}c}
      & \bf    Yuyang Rong &
   \\         &    ShanghaiTech University, China    & 
   \\   & rongyy@shanghaitech.edu.cn
   \\  &  &  
   \\  &  & 
  \end{tabular}
  
    \begin{tabular}{c@{\hspace{1cm}}c@{\hspace{1cm}}c}
 \bf  Hao Wang  & \bf    Jiashan Wang &\bf Hudie Zhou
   \\      ShanghaiTech University   &     University of Washington    & ShanghaiTech University
 \end{tabular}
 
 \vspace{6em}
 2018, US
 \end{table}

\end{frame}

% Table of contents
\begin{frame}{Outline}
  \tableofcontents
\end{frame}
%**************

%**************************
% Section 1. Introduction
%**************************

\section[Introduction]{Introduction}

\begin{frame}[c]{Introduction}
\vfill
\begin{itemize}
\item   Nonlinear Programming
\[
\begin{array}{ll}
\mbox{minimize}&f(x)\\ 
\mbox{subject to}&c_i(x)= 0 ,  \quad i\in\Ecal\\
& c_i(x) \le 0, \quad i\in \Ical
\end{array}
\] 
\item In the  penalty (merit) function 
$$\begin{aligned} \phi(x;\rho)  &= {\color{blue} \rho } f(x)+v(x) \\
v(x) &= \sum_{i\in\Ecal\cup\Ical} v_i(c_i(x)) \\
v_i(z) & =   \left| z \right|,\  i\in\Ecal\quad\text{and}\quad v_i(z) &=  (z)_+,\  i\in\Ical \}
\end{aligned}$$
 \item SLP solves a sequence of LP subproblems to get steps
 \item Penalty-SLP uses penalty (merit) function to select stepsize
 \begin{equation}
\begin{aligned} 
l(d; \rho, x^k) & := \rho \nabla f(x^k)^Td + \sum_{i\in\Ecal\cup\Ical} v_i( c_i(x^k) + \nabla c_i(x^k)^Td)\\
% \tag{\mbox{$\mathscr{P}$}}
   d^k & := \arg\min_{d \in X}\ l(d; \rho_k, x^k)\ \ \text{for some}\ \ \rho_k \in (0,\rho_{k-1}] \\
% \tag{\mbox{$\mathscr{D}$}}
   \max_{\lambda_{\Ecal} \in \mathbb{B}, \lambda_{\Ical}\in\mathbb{B}_+}\ p(\lambda; \rho, x^k) & :=  -  \delta    \|\rho \nabla f(x^k)+\sum_{i\in\Ecal\cup\Ical}\lambda_i \nabla c_i(x^k) \|_* + c(x^k)^T\lambda
\end{aligned}
\end{equation}
\item How to update this $\rho$ at every iteration? 
\end{itemize}
\end{frame}


  



\section[Novelties]{Novelties}

\begin{frame}[c]{Novelty I: Inexact LP solves}

\vfill 

{\red  \leftpointright Proposed: } 
Our proposed SLP technique is specifically designed to be effective in large-scale settings, allowing {\red inexactness} in the subproblem solves. 

\vfill

\begin{figure}[H]
\includegraphics[scale=0.15]{pic/inexact}
\end{figure}
\end{frame}


\begin{frame}[c]{Novelty II: One inexact LP solve to update penalty parameter}

\vfill

{\blue \leftpointright Traditionally: }

\vfill

\begin{figure}[H]
\includegraphics[scale=0.15]{pic/rhoupdate}
\end{figure}

\end{frame}




\begin{frame}[c]{Novelty II: One inexact LP solve to update penalty parameter}

\vfill

{\red  \leftpointright Proposed: } 
Dynamic penalty parameter updating strategy to be employed within the subproblem solve. Not having to accurately solve multiple LPs in a single iteration.

 
\vfill

\begin{figure}[H]
\includegraphics[scale=0.15]{pic/dynamicupdate}
\end{figure}

\end{frame}





\begin{frame}[c]{Novelty III: Automatic Infeasibility Detection}

\vfill

{\blue 
\leftpointright Traditional approach:}  Feasibility Restoration Phase

\vfill

\begin{figure}[H]
\includegraphics[scale=0.15]{pic/feasibilityRes}
\end{figure}

\end{frame}


\begin{frame}[c]{Novelty III: Automatic Infeasibility Detection}

\vfill

{\blue 
\leftpointright Traditional approach:}  Feasibility Restoration Phase

\vfill

\begin{figure}[H]
\includegraphics[scale=0.15]{pic/feasibilityRes2}
\end{figure}

\end{frame}


 


\begin{frame}[c]{Novelty III: Automatic Infeasibility Detection}

\vfill

{\red 
\leftpointright Proposed:} Ensuring that each computed step predicts progress toward minimizing penalty function and constraint violation,   allowing for rapid {\red infeasibility detection.}

\vfill

 \begin{figure}[H]
\includegraphics[scale=0.15]{pic/autoInf}
\end{figure}

\end{frame}

 

 
 

\begin{frame}{Background: Proposed updating strategy} 
Dynamically update $\rho$ {\color{red} within} the LP solver
\[
\begin{array}{ll}
 \mbox{minimize}&  \rho \nabla f(x^k)^Td +e^T(r+s) +  e^Tt \\ \\
 \mbox{subject to}&  c_i(x^k) + \nabla c^i(x^k)^T d =  r_i - s_i ,  \quad i\in\Ecal\\
          & c_i(x^k) + \nabla c^i(x^k)^T d \le t_i, \quad \qquad i\in \Ical\\
          & -\delta e \le  d \le \delta e, \   (r,s,t) \ge 0
\end{array}
\]

\vfill

So that after solving LP, we have

\smallskip

\begin{itemize}
\item {\color{red} A search direction $d$}
\item {\color{red} An appropriate value of $\rho$}
\end{itemize}

\end{frame}














\section{Proposed approach: a dynamic updating strategy}


\begin{frame}{A dynamic updating strategy}
Consider a LP solver for 
\[
l(d; \rho, x^k) = \rho \nabla f(x^k)^Td + \sum_{i\in\Ecal\cup\Ical} v_i( c_i(x^k) + \nabla c_i(x^k)^Td)
\]

While solving the LP,
\begin{enumerate}
\item {\color{blue} When} to update $\rho$?
\item {\color{blue} When} do you think $\rho$ is too large?
\end{enumerate}

 \vfill
 
 \begin{itemize}
\item \underline{Feasibility LP }
\[
l(d; 0, x^k) = \sum_{i\in\Ecal\cup\Ical} v_i( c_i(x^k) + \nabla c_i(x^k)^Td)
\]
gives a descent direction $d$ for $v(x)$

\vfill

\item \underline{Optimality LP}
\[
l(d; \rho, x^k) = \rho \nabla f(x^k)^Td + \sum_{i\in\Ecal\cup\Ical} v_i( c_i(x^k) + \nabla c_i(x^k)^Td)
\]
gives a descent direction $d$ for $\phi(x;\rho)$

\vfill


\item Sufficiently small $\rho$, a descent direction for {\red  both}
\end{itemize}

%we don't want rho to the end
%we don't rho updated too aggressively 

\end{frame}



 


\begin{frame}{$\rho$ update: checking optimality v.s. feasibility}
\begin{itemize}
\item Primal value of optimality LP: $l(d;\rho)$
\item Dual value of optimality LP: 
$$ p(\lambda; \rho, x^k) =  -  \delta    \|\rho \nabla f(x^k)+\sum_{i\in\Ecal\cup\Ical}\lambda_i \nabla c_i(x^k) \|_* + c(x^k)^T\lambda
$$

\end{itemize}

\vfill


\begin{minipage}[c]{.6\textwidth}

\begin{itemize} 
\item Solving   optimality LP, duality gap 
$$ 0\le l(d^{(j)}; \rho) - p(\lambda^{(j)}; \rho)   \to 0$$
 gap ratio
$$r^{(j)}_{opt} := \frac{l(d^{(0)};\rho) - l(d^{(j)};\rho)   }{l(d^{(0)}; \rho) - p (\lambda^{(j)};\rho)  } \to 1$$
 \item Similarly, solving  feasibility LP, gap ratio
$$r^{(j)}_{fea} := \frac{l(d^{(0)};0) - l(d^{(j)};0)   }{l(d^{(0)}; 0)- p(\lambda^{(j)};0)  }  \to 1 $$
\end{itemize}

\end{minipage}
\begin{minipage}[c]{.35\textwidth}
\begin{figure}[htbp]
\centering
\includegraphics[scale=0.25]{pic/primaldual}\\
\end{figure}

\end{minipage}

\end{frame}






\begin{frame}{$\rho$ update: checking optimality v.s. feasibility}

\begin{itemize}
\item   Choose tolerance $\omega>0$
\item  As $j\to \infty$, still have 
$$r^{(j)}_{opt} := \frac{l(d^{(0)};\rho) - l (d^{(j)};\rho) + \omega  }{l(d^{(0)}; \rho) - p (u^{(j)};\rho) + \omega } \to 1 $$

But not necessarily 
\[
   r^{(j)}_{fea} := \frac{l(d^{(0)};0) - l (d^{(j)};0) + \omega}{l(d^{(0)}; 0)- p(w^{(j)};0) + \omega}  \to 1
\]
if $\rho$ is not chosen properly. 
\end{itemize}
\vfill 

If 
%\begin{equation}
%\tag{OPT}
%\label{red.penalty}
\[ {\color{red} r^{(j)}_{opt}  \ge \beta_{opt}} \]
% \end{equation}
with $\beta_{opt} \in (0,1)$ holds, but  
%\begin{equation}
%\tag{FEA}
%\label{red.fea} 
\[ {\color{red} r^{(j)}_{fea} \ge \beta_{fea} } \]
%\end{equation}
with $\beta_{fea}    \in (0,\beta_{opt} )$ doesn't hold... then $\rho$ is large

\vfill

%Updating strategy:  
%
%$$
%\begin{cases} \text{If OPT {\color{red}holds}, FEA {\color{red} doesn't hold}}  &  \text{update  } \rho_{(j+1)} \gets \rho_{(j)}/2 \\
%\text{If OPT and FEA {\color{red}both hold}}  &  \text{terminate QP solver}
%\end{cases} 
%$$


\end{frame}

%
%\begin{frame}{$\rho$ update: checking optimality v.s. feasibility}
%
% Why FEA and OPT?   After solving the QP, we have 
%
%
%\begin{itemize}
%\item {\color{red} A search direction $d$}
%\item {\color{red} An appropriate value of $\rho$}
%\end{itemize}
%
%\vfill
%
%so that 
%\begin{align*}
%\Delta J(d^k;x^k, \rho) & \ge \beta_{opt} \Delta J(d^*; x^k,\rho)\\
%\Delta J (d^k; x^k, 0 )  & \ge \beta_{fea} \Delta J(d^*; x^k, 0 ) 
%\end{align*}
%
%\vfill
%
%Then go to line search, just one QP is solved per iteration!
%
%\end{frame}
%



\begin{frame}{$\rho$ update: checking Complementarity}

Define: 

\[ \begin{aligned}
\Ecal_+(d) & := \{  i\in \Ecal:  \langle a^i, d\rangle + b_i > 0\}\\
\Ecal_-(d) & := \{ i\in\Ecal:  \langle a^i, d\rangle + b_i < 0\} \\
\Ical_+(d) & := \{i \in\Ical: \langle a^i, d\rangle + b_i > 0\}
\end{aligned}
\]

Observe: 

\[ \chi(d,\lambda) = \sum_{i\in \Ecal_+\cup \Ical_+} (1-\lambda_i) |\langle a^i, d\rangle + b_i| + \sum_{i\in\Ecal_-} (1+\lambda_i) |\langle a^i, d\rangle + b_i|  \]

By complementarity,    {\blue $\chi^{(j)}: =  \chi(d^{(j)},\lambda^{(j)}) \to 0$  as $j\to 0$ }
 
 \vfill
 
 
 Require \[ \chi^{(j)} \le (1-\beta_c)^2 (v(x_k)+\omega), \]
 or equivalently
 {\red \[ r_c^{(j)}: =1- \sqrt{  \frac{ \chi^{(j)} }{v(x_k)+\omega} } \ge \beta_c \]}
 
\end{frame}




 




\begin{frame}{$\rho$ update:  DUST} 


\underline{ A {\blue D}ynamic {\blue U}pdating {\blue ST}rategy (DUST):} 

\vfill 



Given $\rho_{(j)}$ and the $j$-th iterate $(d^{(j)}, \lambda^{(j)})$, perform the following:

\begin{itemize}
\item  If $r_{\phi}^{(j)} \ge \beta_\phi$, $r_c^{(j)} \ge \beta_c$ and $r_v^{(j)} \ge \beta_v$ hold, then terminate; 
\item  else $r_{\phi}^{(j)} \ge \beta_\phi$, $r_c^{(j)} \ge \beta_c$, but $r_v^{(j)} \ge \beta_v$ does not, then set $\rho_{(j+1)} \gets \rho_{(j)}/2$
\end{itemize}
  
  \vfill
 
 
  {\blue 
\leftpointright Intuition:}  

 \begin{itemize}
 \item Whenever optimality and complementarity are sufficiently improved,  tune  $\rho$ to guarantee improvement on feasibility 
 \item Shouldn't place too much emphasis on optimality and complementarity
 \end{itemize}
 
\end{frame}





\begin{frame}{$\rho$ update:  PSST} 

\underline{ A {\blue P}osterior {\blue S}ubproblem updating {\blue ST}rategy (PSST) }

\vfill

At the end of the LP solve, check whether 
$${\color{blue}   \Delta l(d; x^k,\rho)  + \omega_k  \ge  \beta_l [\Delta l(d; x^k, 0) + \omega_k}] $$
holds.  If not, reduce $\rho$ to make it satisfied, e.g., 

\[ \rho_k \gets \frac{(1-\beta_l)( \Delta l(d; 0;x^k)+\omega_k) }{ \nabla f(x^k)^Td^k  } \]
 or smaller. 

\vfill

  {\blue 
\leftpointright Intuition:}  Too much emphasis on feasibility?
\end{frame}


\begin{frame}{Parameter requirement}

Require: 
\vfill

\begin{itemize}
\item Tolerance should get tighter and tighter
\[ \boxed{\omega_k \to 0 }\]
\item Priority is on optimality 
\[ \boxed{1> \beta_\phi  > \beta_v > 0}\]
\item Complementarity should be satisfied roughly 
\[\boxed{ \beta_c \in (0,1)}\]
\item Shouldn't place too much emphasis on feasibility 
\[\boxed{ \beta_\phi(1-\beta_v)   > \beta_l>0 }\]
\end{itemize}
 \end{frame}






\section[Convergence Analysis]{Convergence Analysis}




\begin{frame}{Finite updates per subproblem}

\begin{assumption}[Subproblem Assumption]\label{ass.algorithm}
Functions $f$ and $c_i$ for all $i \in \Ecal\cup\Ical$, and their first- and second-order  derivatives, are all  bounded in an open convex set containing $\{x^k\}$ and $\{x^k+d^k\}$, i.e., there exists a constant  $\kappa_1>0$  such that  for any $x$ in this set, 
\[  \rho_0\|\nabla^2f(x)\|_2 \le \kappa_1\ \text{  and  }\  \|\nabla^2c_i(x)\|_2 \} \le \kappa_1, i\in\Ecal\cup\Ical.\] 
Defining the index set 
\[ \mathcal{U} := \{k \in \mathbb{N} : \mathcal{R}^k\ne\emptyset \} . \]
Moreover, for every $k\in\mathcal{U}$, let $j_k$ be the subproblem iteration number corresponding to the value of the smallest ratio $r_v$, i.e.,
\[   r_v^{(j_k)} \le r_v^{(i_k)} \quad \text{for any }\  i_k\in\mathcal{R}^k.\]
Also, defining the index set 
\[  \mathcal{T} := \{ k \in\mathbb{N}: \rho_k \  \text{is reduced by}\ \eqref{psst} \}. \]
From these definitions, it follows that $\rho_k < \rho_{k-1}$ if and only if $k\in\mathcal{U}\cup\mathcal{T}$.
\end{assumption}

\vfill

\begin{theorem}
Given $\nabla f(x^k), \rho_k$ and set $C$.  Consider a QP solver such that the Subproblem Assumption  is satisfied.   There exists $\rho_*>0$ such that for any $\rho\in(0,\rho_*)$,  if the algorithm is employed, then the update strategy is never triggered. 
\end{theorem}

\end{frame}



\begin{frame}{Convergence of SLP}
\bassumption
  There exist a compact convex set $X \subset \mathbb{R}^n$ with $0 \in \interior{X}$ and positive scalar constants $\underline\Lambda_0$, $\overline\Lambda_0$, $\underline\Lambda_\rho$, $\overline\Lambda_\rho$, and $K_0$ such that the following hold true.
  \benumerate
    \item[(i)] $f$ and $c_i$ for all $i \in \{1,\dots,m\}$, and their first- and second-order derivatives, are all  bounded in an open convex set containing $\{x^k\}$ and $\{x^k+d^k\}$.
    \item[(ii)] Positive definiteness of the (approximate) Hessians. There exist constants $\overline\Lambda \ge \underline\Lambda > 0$ such that for all $k\in\mathbb{N}$ and  any $\rho\in[0, \rho_0]$,  
    \[ 0< \underline\Lambda \le \underline\lambda_{0,k} \leq \overline\lambda_{0,k} \leq \overline\Lambda\quad \text{and}\quad 0 < \underline\Lambda \leq \underline\lambda_{\rho,k} \leq \overline\lambda_{\rho,k} \le \overline\Lambda.\]
    \item[(iii)] Uniform boundedness of the derivatives. 
  \eenumerate
\eassumption

\vfill

\begin{theorem}  Suppose Assumption  holds. Then, one of the following occurs.
\begin{enumerate}
\item[(i)]  $\rho_k \to \rho_*$ for some constant $\rho_*>0$ and each limit point of $\{x^k\}$ either corresponds 
to a KKT point or an infeasible stationary point.
\item[(ii)] $\rho_k \to 0$ and all limit points of $\{x^k\}$ are infeasible stationary points.
\item[(iii)]  $\rho_k \to 0$, all limit points of $\{x^k\}$ are feasible, and the MFCQ fails to hold at all limit points 
where $\rho$ is updated. 
 
\end{enumerate}
 
\end{theorem}

\end{frame}






 






\section{Preliminary numerical results}

 
  
\begin{frame}{Experimental settings}

\begin{itemize}
\item Test set:  126 CUTEr Hock-Schittkowski (\texttt{hs}) problems
\vfill
\item Subproblem algorithm: Primal Simplex
\vfill
\item Parameter selection:   
\begin{table}[H]
\centering
\caption{Parameters in FoNCO}
\label{tab.para}
\begin{tabular}{c|ccccccccc}\hline
 Parameter  & $\rho_0$   &  $\beta_\alpha$  &  $\beta_v$& $\beta_\phi$ & $\beta_l$                  &  $\gamma_0$  &  $\theta_\gamma$   & $\theta_\alpha$    & $\delta_0$    \\ \hline 
 Value   &   $1$    &  $10^{-4}     $  &  $0.3    $& $0.75$       & $0.135$ &     0.01     &  0.7   &   $0.5$  & 1  \\  \hline\hline

Parameter &      $\bar\sigma$     &  $\underline{\sigma}$  &  $\delta_{\min} $  &    $\delta_{\max}$      & $\texttt{Iter}$ & $\texttt{Iter}^{ps}$  &  & & \\  \hline
 Value &           $0.3$  &   $0.75$  &  $64$       &   $10^{-4}$  &        $1024$   &  $100$ & & & \\  \hline
\end{tabular}
\end{table}
\item Termination 
\begin{equation}\label{relative.kkt}
\begin{aligned}
  & \epsilon_{kkt} := \frac{\max(E_{opt}(x^k, \lambda^k, \rho_k), E_c(x^k, \lambda^k))}{\max(1, E_{opt}(x^0, \lambda^0, \rho_0), E_c(x^0, \lambda^0))}\\
  & \epsilon_{kkt} < 10^{-4} \\
  & v(x^k) < 10^{-4}
\end{aligned}\end{equation}

\end{itemize}

\end{frame}


\begin{frame}{Numerical Results}


Observations: 
\vfill
\begin{itemize}
\item  
Successfully solved $116$, success rate   $  = 113/126 \approx 90 \%$. 
\vfill
\item   
We noticed that some problems
 are sensitive to the selection of trust region radius. For examples, problems HS87, HS93 HS101, HS102 and HS103 failed with initial trust region radius $1$. We re-ran those 5 problems with a smaller initial trust region radius $\delta_{0} = 10^{-4}$. All these 5 problems are solved successfully. We believe the robustness of our proposed algorithm could be improved with a more sophisticated trust region radius updating strategy. 
 \item Simplex method employed in our implementation is very efficient. For most of the problem, very few number of pivots is needed for each subproblem solve. 
%We notice that the size of the positive eigenvalues of these problems are around $10^{-4}$, so our modification significantly changed the structure of the Hessian preventing convergence within the iteration limit.
\end{itemize}



\end{frame}

 

\begin{frame}[c]{Numerical Results:  final $\rho$   }

\begin{figure}[H]
\includegraphics[scale=0.15]{pic/rho}
\end{figure}

\end{frame}

 


\begin{frame}[c]{Numerical Results:  Number of Iterations and Function Evaluations}

 \includegraphics[scale=0.13]{pic/NumIter}
  \includegraphics[scale=0.13]{pic/FunEval}
 
\end{frame}

 



\begin{frame}[c]{ }

\centerline{\red \huge Thank you!!}

\vspace{3em}

\centerline{\red \huge  Contact me: rongyy@ShanghaiTech.edu.cn}

\vspace{3em}


%\centerline{\red Find my preprint paper: \underline  {  https://arxiv.org/abs/1803.09224}}

%\vspace{3em}


\centerline{\red Find my code: \underline  {  https://github.com/DataCorrupted/FoNCO-ps}}



\end{frame}

 



%
%
%
%\section{Huge scale setting}
%
%\begin{frame}{Huge scale setting}
%\begin{itemize}
%\item Our numerical experiment is a Limited-Memory BFGS (L-BFGS)
%
%\vfill
%
%\item The proposed updating strategy can maintain the structure of Hessian low-rank decomposition (separate approximation for two Hessians) : 
%
%\begin{eqnarray*}
%H(x^k,\lambda^k,\rho_k)  &=&\sigma  I+\Psi \Sigma^{-1}\Psi^T,\\
%H(x^k, \lambda^k, 0 ) &=&\gamma I+\Phi \Gamma^{-1}\Phi^T,
%\end{eqnarray*}
% \item  By Sherman Morrison formula
% \begin{align*}
%H_0^{-1}     & = \frac{1}{\gamma}\left[ I - \Phi(\gamma\Gamma+\Phi^T\Phi)^{-1}\Phi^T \right] \\
%H_\rho_{-1}      & = \frac{1}{\sigma}\left[ I - \Psi(\sigma \Sigma + \Psi^T\Psi)^{-1} \Psi^T  \right] \\
%H_{\rho,\mu}^{-1} & =  \frac{1}{\sigma+\frac{m}{\mu}}\left[ I - \Psi[(\sigma +\frac{m}{\mu})\Sigma + \Psi^T\Psi]^{-1} \Psi^T  \right].
%\end{align*}
%\end{itemize}
%\end{frame}
%
%
%
%
%
%
%
%\begin{frame}{Huge scale setting}
%
%\begin{itemize}
%\item After reducing $\rho$ to $\bar\rho$, the decomposition can be easily computed: 
%\begin{align*}
%H_{\bar \rho} & =  \bar\rho H_f + H_0 \\
%& = \frac{\bar \rho}{\rho} ( H_0+\rho H_f) + (1-\frac{\bar\rho}{\rho}) H_0 \\
%& =\tau H_\rho + (1-\tau) H_0\\
%& = \bar\sigma I + \tau \Psi\Sigma^{-1}\Psi^T + (1-\tau) \Phi \Gamma^{-1} \Phi^T\\
%& = H_\tau + (1-\tau)\Phi\Gamma^{-1}\Phi^T,
%\end{align*}
%\item  By Sherman Morrison formula
%\begin{align*}
%H_{\bar\rho}^{-1} & = H_{\bar\tau}^{-1} - H_{\tau}^{-1} \Phi \Theta_4^T H_{\tau}^{-1}\\
%H_{\tau}        & = [\tau\sigma+(1-\tau)\gamma] I + \tau \Psi\Sigma^{-1}\Psi^T  
%\end{align*}
%\end{itemize}
%
%
%\end{frame}
%
%
%
%\begin{frame}{Future work}
%
%\bite
%\item Incorporate proposed updating strategy into a SQP framework
%\item Prove ``uniform boundness" for SQP algorithm
%\eite
%
%\end{frame}
%
%
%
%\begin{frame}
%
%\center{\Huge \color{red} \bf Thank you!}
%
%\end{frame}
%
%
 

\end{document}


